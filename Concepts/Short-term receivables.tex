% !TeX spellcheck = en_GB
\documentclass[10pt,paper=a4,parskip]{scrartcl}
\usepackage[utf8]{inputenc}
\usepackage{amsmath}
\usepackage{amsfonts}
\usepackage{amssymb}
%\usepackage{draftwatermark}https://codeyarns.com/2010/05/05/how-to-add-draft-watermark-in-latex/
\usepackage{graphicx}
\usepackage[hyphens]{url}
\author{Martin Weik}
\title{Case: Short-term receivables}
\newif\ifforme
\formetrue
\formefalse
\newcommand{\note}[1]{\unskip\ignorespaces\ifforme\textbf{#1}\else{}\fi}

\begin{document}

\maketitle
\begin{abstract}
This case study investigates the proper handling of short-term receivables (and liabilities) in UGB, IFRS and the resulting mapping in REA as Financial Resource and the generic ``original'' REA.
These questions arise:
\begin{enumerate}
	\item Short-term receivables (and liabilities) are booked in analogy to fixed-interest loans.
	\item Arising accruals are booked as agios?
	\item Turnover is booked with 100\%?
\end{enumerate}
\end{abstract}

\section{Motivation}

\bibliographystyle{alpha}
\bibliography{../References/References}

\end{document}