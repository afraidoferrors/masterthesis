% !TeX spellcheck = en_GB 
% !TEX root = Thesis 
\chapter{Introduction to REA}\label{chap:IntroREA}


\section{State of the art REA}
\subsection{REA}
What is
\subsection{Extension through policies}
Types, Commitments
\subsection{Summary of Otology paper}
Types of connections, tasks


\section{The purpose of financial Accounting}
After Horngreen, Accounting in general is the profession executed by Accountangs to \enquote{design their system after considering the types of information desired and other users}.\cite{horngren1984introduction}[p.3]
The desired informations are commonly used to answer questions after the perfomance of a \enquote{organization} in a given period and the (financial) standing in a certain moment in time.\cite{horngren1984introduction}[p.3]

Ijiry (as additional source) states that \enquote{Accounting is a system for communicating the economic events of an entity}. \cite{Ijiri1967}[p.3]
That is \enquote*{The economic events of an entity must be represented by an organized set of symbols which are suitable for communication}. \cite{Ijiri1967}[p.3]

The most resulting artifacts of financial accounting activity are the income statement and the balance sheet.
\enquote{To obtain these statements} 


\begin{figure}
	\centering
	\caption{Fundamental relationsships in Accounting}
	\label{fig:accounting-fundamental-relationsships}
	\includegraphics[width=0.7\linewidth]{"../figures/umlet/02-Process_of_Accounting_Horngreen"}
	\caption*{After cite{horngren1984introduction}[p.3]: }
\end{figure}

\section{Event recognition}





This thesis starts with a thorough discussion of important concepts in context of REA.








\section{The relevance of Events in Financial Accounting}
\section{Accounting measurement and event recognition}
% Process and measurement of Accounting in general



The entity represents the horizon for events that need to be recognized.
All events that impacts the financial position 
Resources that are under control...

First of all it might be beneficial to recall that all kinds of accounting rely on the fact that events are recognized and recorded.
In 


For financial accounting 

\cite{horngren1984introduction} 


\section{The entity as a subject of Accounting}

\section{The process of acknowledge events and measurment}


\section{Introduction - The roots of financial accounting}
\section{Short and long living companies}
\section{Profit \& Loss and the concept of accrual}
\section{The balance sheet}


\section{Causal accounting}

2021-09-12

This thesis considers two approaches to the basics of financial accounting.

+++++++

Excerpt: How can the consumption of services be modelled?

Intro: Horngreen argues that every purchase of service as expense goes through balance sheet:
- Purchase on Assets for immediate consumption
- Consumption of service as expense
=> It is a kind of depreciation

In REA the first booking is easy:
Resource Money is exchanged against resource "Service".
But then this "Service" has to be consumed against what?

But: The benefit of "Service" stays inside the company, is it a kind of depreciation?
So this could be modeled as a pure financial event.

+++++++



The understaning of doing business and measuring the success has drastically changed over the past years.
Companies are required to deliver financial information at any time.
Thinking of Italian traders in the medieval age the motivation to account business transactions then today:
Back then it was usual to not have an annual closing, they rather calculated the success of a long lasting trade trips on a per trip base over years as Eugen Schmalenbach states:

\blockquote[\cite{schmalenbach1962}][]{Hier das Zitat!!!}

The trader did business on his own and a sum up was done after all related transactions where finished.
So there was no need to distungish between the owne and the company and also no need for accruals.

Schmalenbach starts with a hyphotetic business that lives only for a single purpose and is rolled up after this purpose is over.
The business gets credit from banks 
Within the lifetime of this company only incomes and expenses measured and at the end a single balance sheet is created:
All 
This balance sheetare aufgelöst after the purpose of this company i

Unterscheidung Aufwand und Ausgaben


Financial accounting took a long way from medeval ages where the economic success was calculated by economic long term project there was no difference between. Success was calculated after a cruise accross the ocean was finished on per project base, a yearly financial statement was unkown and not in focus.

Time after time and as money markets arised it was more and more important to give a complete view.


Following these requirements the P \& L oriented structured balance sheet 

A financial statement has to fullfill many requirments: It should give Investors an insight

Over time a company is considered as independent 


Companies and economic activites have evolved from personal inverstments where there was no seperation between person and company and where the 

Development of financial accounting... in respect of target audience: Creditors, Owners, 





Financial Accounting is done with double entry accounting since the 15th century \todo{Citation} and the main principle has hardly changed thereafter.







As kind of grounding of this thesis foundations of Accounting are investigated in this chapter.
The starting point is an overview over the process of accounting and fundamental terms of accounting, the main source for that part will be \cite{Horngren1984}.
Then fundamental connections between events, resources and following implications for measurment, classification and valuations are considered, this part will heavily rely on \cite{Ijiri1967}.
At last IFRS as concrete standard for measurement and classification is introspected as this will be the basis for mapping activities between the REA Data Model and financial accounting artefacts.
starting with the process of accounting, going further to typical artefactes and an analysis of mechanics in accounting.

The well known double entry accounting with it's first written introduction in \ldots by \ldots is widely anticiptated. % and worldwide used.
\enquote{Accounting is the major means of communicationg about the financial impact of an organzation's activities.}\cite[p.3]{Horngren1984}
It's
It's main artefacts are t-accounts, journals and as core the so called ALE Equation.

But what is behind the scenes?


To make rational choices, comprehensive information about a companies wealth to interested actors / audience is crucial.
Financial Accounting supports these decisions by delivering an overview over a companies assets and liabilities and Accrual Accounting extends this support by making the companies performance in a given period visible.

This chapter introduces fundamentals of Accounting and presents the audience and their benefits of the accrual accounting artefacts and relates fundamental concepts of accrual and further IFRS accounting to REA\textcopyright accounting.
% The outcome will be an answer, if the REA Accounting Model is feasable to 

Scope: Financial Accounting -> financial reporting: e.g. values of assets, cash, income-statement

Financial Reporting as statutary reporting

Historical: Accounts where closed when , there where 

We mean financial reporting as statutory reporting as it is needed to inform Owners, Creditors and tax authorities that has to be done at least once a year and structured as defined by laws or regularities

History of accounting\dots
Balance sheet was rather a Reume over projects (like sailings) rather than a regulary executed task.
Often a balance was taken when the main journal was full and a new book was startet.

Intention of balance sheet:
Overview over capital
Calculation of Earnings in a period
Schmalenbach calls this a dynamic balance, where in dopik the Profit \& Loss account is embeded between to balances.

Historicaly there was a lot of discussion about valuation of items, higher taxrates e.g. yield to a motiviation to minimize wealth and financial result to tax optimization.
The only intention of analysis financial accounting in this thesis is solely technical based.

There are different kind of events possible: once that are relevant for the business but not for finincial reporting, events that are relevant also for financial accounting and events that only exist to have propber visible accounting e.g. events that are deducting assets or to do the annual closing.



Start with cash -> then accrual


% \enumerate{}
Stake holders, why do we do accounting: Investors
Accounting terms: Entity, Transaction, Valuation, Measurement
Transaktion starts with events.
Events can be relevant for Business, or have a financial impact (beispiel ijiri manager)
Duration of events (ijiri?)

\section{The process of accounting}
Process of accounting: recognizing an event relevant for financial accounting, valuation

\begin{figure}
	\centering
	\caption{Process of Accounting}
	\label{fig:accounting-process}
	\includegraphics[width=0.7\linewidth]{"../figures/replace/CollaborativePerspective"}
	\caption*{Taken from \cite[p.3]{horngren2006introduction}. This figure }
\end{figure}

\section{The entity as core concept}
Important: The company itself is the point of view, not
Imporatant is to distinguish between a company and its owner. In financial accounting, the company is the legal entity and therefore defines the point of view and the owner has rights on the companies assets.

\section{events}

\section{State of the art REA}
\subsection{REA}
What is
\subsection{Extension through policies}
Types, Commitments
\subsection{Summary of Otology paper}
Types of connections, tasks
\section{Foundations of Accounting}
REA itself is an accounting model but let's recap what the most important
properties of accounting are. \todo{citation! to REA Paper}

These foundations of accounting provide the grounding for the following check of common properties and possibilities for wmapping.

\subsection{The Entity}
The entity represents the event horizon of an Financial Accounting System.
An Entity can be a company, a department or also a group of companies and an Financial Accounting system must be assigned to exactly one.
Everything that touches the financial positions of an corresponding (zugehörig) entity is recorded.
Everything else is ignored.\cite[p.15]{horngren1984introduction}

If an employee gets sick that possibly leads to damage (e.g. by delaying projects) to the entity but this event will not yet find it's way to a Financial Accounting System as he will be receiving is usual salaries.
For paid leaves there is a need to build a provision in case the employee leaves the entity and can not consume his quota.
So when he is going on holidays this will be reflected in Financial Accounting as the quota is reduced.

REA itself was also designed as entity centric accounting system but then extended to an \enquote{Open-edi Business Transaction Ontology (OeBTO)}\cite[Figure 2]{ISOIEC1594442015}\ref{fig:collaborative-perspective}.
Although the concepts of REA allow a wider view on Accounting systems in this thesis the \enquote{Trading Partner View} is used as there exists no collaborative spac in Financial Accounting Systems and to make the mapping between Financial Accounting and REA Accounting simpler and support better understanding.
As preview, it is preferable that events relevant for Financial Accounting are depicted either as \enquote{credit} or \enquote{debit} as shown in \cite{schwaiger2015aleandrea} and make sense only from the entity's point of view.

% \begin{figure}
% 	\centering
% 	\caption{REA Collaborative Perspective}
% 	\label{fig:collaborative-perspective}
% 	\includegraphics[width=0.7\linewidth]{"../figures/replace/CollaborativePerspective"}
% 	\caption*{Taken from \cite[Figure 2]{ISOIEC1594442015}. This figure demonstrates the two possible point of views that can be used. In this thesis the collaborative space is not relevant therefore the \enquote{Trading Partner View} is used.}
% \end{figure}

\subsection{Purpose of accounting}
Audiences
Scope of accounting: Support of Decisions, AUD IT GAAP => Tax
\subsection{Recognition of Events}

Accounting is generally speaking the process of recognizing relevant event, to analyse and store them and finally create financial statements that can be interpreted by interested audience.
What kind of events and when they are in scope of finance accounting is

Financial accounting stores only interpreted and aggregated data in monetary units whereas REA recognizes the events with all their attributes itself.
McCarthy calls this the \enquote{essence of events}\todo{citing}.
Figure \ref{fig:EventsTimeOfRecording} shows the process of accounting and compares the moment of recognition through in financial accounting and REA accounting in a timeline.
It must be considered, that not every
\begin{figure}
	\centering
	\caption{Time of recording in Accounting systems}
	\label{fig:EventsTimeOfRecording}
	\includegraphics[width=0.7\linewidth]{"../figures/EventsTimeOfRecording"}
	\caption*{This figure after \cite{horngree1981nintrofinacc} shows the different moments in time when recording happens. Traditional accounting first interprets the event and records the interpretation whereas REA stores all needed information of an event to interpret it afterwords.}
\end{figure}

\subsection{Management and Financial Accounting}
Purpose of accounting: Targeted Audience, difference between management and
financial accounting
\section{Concepts of Financial Accounting}
\subsection{The Statement of Financial positions}
The statement of financial positions is also called \enquote{balance sheet} as this equation has always to be true:
\[ Assets = Equities \]
The Assets consist of resources, that are under the equities control and the Equities can be further distinguished to:
\[ Equities = Liabilities + Shareholders equity\]


Different Types of Assets:

Tangible \& Intangible

Financial Assets

Direct tradable Assets acting as resources: Tangible \& intangible Assets,
Inventories, Cash

Discussion of Claims

Assets have nature of commitmens of other companies for future resource flow

Problem: Commitments can not be exchanged but this is possible with Assets that have only one borrower

\subsection{Liabilities}
Claim not feasable

Commitment

\subsection{The income statement}
Accrual accounting here\ldots

Development of equation

Debit / Credit

Changes in Equity\ldots

Different Types of bookings:
Even Services are in fact

=> Events that have effect to income statement and events that have not.

Accrual Accounting: Revenue and Expense are connected

2 Beispiele: Kauf von Verbrauchsmaterial => 1) Sofortige Abschreibung 2) Kauf auf Lager, dann Verkauf => Muss bei Mapping von Events berücksichtigt werden / Oder hängt es an der Resource? Lagerresource vs. Verbrauchsresource Inventoryresource vs. consumables

\section{Accrual Accounting}
Special ledgers: Creditors, Debitors, Ageing




\section{IFRS Accrual Accounting}
Usual artefacts out of IFRS Accounting: Balance Sheet, Income Statement, Changes in Equity, Cash Flow
Other usefull artefacts: Ageing list by business partner,


More or less interresting for valuation and recognition, not technical
modelling.

Will be used for Case study\ldots

\section{public obligations}
VAT<
\section{Summary and Conclusion}

The elementary concepts are matching very well.
REA accounting was invented with all types of accounting in mind.

\begin{table}
	\caption{}\label{tab:Elementary concepts}\caption*{}
	\begin{tabular}{|p{0.15\linewidth}|p{0.15\linewidth}|p{0.15\linewidth}|p{0.55\linewidth}|}
		\hline
		Categorie & Accrual Accounting & REA\textcopyright{} Accounting & Analysis result
		\\
		\hline
		Scope     & Entity             & REA Internal View              & REA itself can be used from an outside or
		inside entities viewpoint. In this thesis the internal viewpoint is
		choosen
		\\
	\end{tabular}
\end{table}