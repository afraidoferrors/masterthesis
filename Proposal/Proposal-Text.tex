% !TeX spellcheck = en_GB

%this is needed to leave out the top level numbering by chapters for figures in this proposal
%\renewcommand{\thefigure}{\arabic{figure}}
%this controls the width of long captions
\captionsetup{width=0.8\textwidth}



%\todo{Created / Printed: \today}
\chapter{Proposal}
\section{Problem definition}\label{sec:problem}
McCarthy introduced REA with accountants and double-entry bookings in mind: “...time to rethink some of the basic constructs of traditional double-entry bookkeeping.” \cite{mccarthy1982rea}, p.554.
It definitely has it's mares by introducing quantities and connecting events that belong to one business transaction. 

Very important elements for accrual accounting like liabilities, deprecations are mentioned but not fully conceptualized.
The basic model is focusing on realised events and therefore can only handle tangible \& intangible assets and cash as resources that can be traded on a spot-market.
%\todo{non-exchange events, partional }

Elements beyond spot-market are discussed as discrepancies between the view of accountants and non-accountants, e.g. on accruals:
%McCarthy already acknowledges discrepancies between accountants and non-accountants:
“Cases where they probably would disagree would involve accruals such as wage expense or interest revenue.” \cite{mccarthy1982rea}, p. 572.
Even if issues are identified, only different approaches are considered without any final verdict.

Table \ref{tab:accrualaccounting} compares important elements of accrual accounting with the REA accounting model and shows where further investigations are needed.
%how elements of accrual accounting can be mapped already and where further investigation is needed.
%Definitions beyond spot-market elements are vague and 
So it's especially unclear how important statements for accrual accounting can be generated out of a REA based model:
\begin{itemize}[noitemsep]%,topsep=0pt,parsep=0pt,partopsep=0pt]
	\item balance sheet
	\item profit \& loss
	\item changes in equity
\end{itemize}


Although REA evolved more and more to a inter-company modelling language it has still good properties for modelling accrual accounting also.
Target of this thesis is therefore to conceptualize a REA-based accrual accounting model and to show that it covers all key element for accrual accounting.
\begin{table}
	\caption{Accrual accounting and REA Accounting Model comparison}\label{tab:accrualaccounting}
\begin{center}
\begin{tabular}{|p{3,5cm}|p{5,8cm}|p{5,8cm}|}
	\hline 
	Accrual accounting & REA Accounting Model \cite{mccarthy1982rea} & Shortcomings \\ 
	\hline 
	Assets (A) & Tangible and intangible assets are modelled as resources. Receivables are not explicitly covered in the original model; proposed is a claim as “imbalanced” REA model on pages 567f. %\cite{mccarthy1982rea}
%	Later commitments are introduced to model them as future exchange of resources \cite{mccarthy2006reapolicy}
	& An investigation for each type of receivable is needed, if existing elements in REA are sufficient (e.g. ageing). \\ 
	\hline 
	Liabilities (L) & Liabilities are not explicitly covered in the original model; proposed is a claim as “imbalananced” REA model on pages 567f. %\cite{mccarthy1982rea}
%	Later commitments are introduced to model them as future exchange of resources \cite{mccarthy2006reapolicy}
	& An investigation for each type of liability is needed, if existing elements in REA are sufficient (e.g. ageing). \\ 
	\hline 
	Equity (E) & %\cite{mccarthy1982rea} 
	Equity is defined as residual size. It is proposed to identify “equity investments” and “dividend distribution” as explicit events to model the structure of equity on page 575. & This approach leaves out the definition of equity via the changes-in-equity statement which is a very important artifact in accrual accounting. \\
	\hline
	P\&L accounts (E) & %In \cite{mccarthy1982rea}
	 P\&L accounts are mapped to Events. & As events are realisations, future events and connected resource flows need to be discussed. \\ 
	\hline 
	Depreciations (E \& A) & On page 574 modelling as “related” connections via “macro-level duality” links to multiple corresponding materialized events is proposed. & As depreciations are usually loosely coupled to specific events a more generic approach is needed. \\ 
	\hline
	Provisions (E \& L)  & Provisions are not explicitly covered in \cite{mccarthy1982rea}. & As provisions are future events with a certain probability without underlying contract it's not yet defined how to model them with existing elements of the REA accounting model. \\
	\hline
%	stock revaluation
%	& under deprecation?? &
%	\\
%	\hline
	Accruals ( A or E \& L ) & These are mentioned as “approximate partitions of exchange transactions” on  page 572 and no definite conclusion is done. & In accrual accounting it is sometimes necessary to recognize expenses before the related exchange event has happened. A thorough discussion is needed. \\ 
	\hline 
	Derivative financial instruments ( A \& E \& L )& Financial instruments are not covered. & 
%	As this topic is not covered by REA a deeper discussion is necessary
	\cite{schwaiger2015aleandrea} shows that especially financial investments can be modelled in a good manner. Still a detailed discussion will be beneficial.
	\\
	\hline
	... & ... & ... \\
	\hline
\end{tabular} 
\end{center}
	\caption*{This table shows accrual accounting and their respective counterpart in the REA accounting model.
	Referenced pages point to pages in McCarthy's Paper \cite{mccarthy1982rea}.
     The characters in paranthesis depict if an element affects Assets (A), Liabilities (L) or Equity (E)
	 As can be seen further investigation is necessary. The three dots in the last line indicate that more elements than in this table will need consideration.}
\end{table}




%\subsection*{Drawbacks of traditional Information Systems}

%Problem statement - existing accounting systems are cumbersome
%Business Information Systems are commonly used to store data that are shared between different departments like HR, Sales, Production, Accounting etc.
%If a company uses separate systems this means often incompatible ``data-islands'' and that leads to a lot of adapting tasks to get data from one system, sanitize and import them into another system.
%Data in separated systems are duplicated by design and data synchronization tasks have to be actively maintained.
%ERP systems only solve parts of this problems because they usually consist of more or less integrated subsystems that exchange data internally.
%This architecture leads to duplicates again and if data get corrupt it's a painful task to restore a working state.

%%% nice for the thesis, to much for the proposal
%\cite[pages 2-6]{dunn2005enterpriseinfosys} argues that specialized departments inside a company tend to build their own ecosystem of applications after their needs that don't interoperate with systems from other departments by default.
%If not taken care processes are likely to focus only on the ``own'' department instead of being integrated to the companies holistic view.
%This leads almost always to non-integrated systems that have to run besides the ERP system, receive or send data and there is always a need to maintain ETL-tasks for proprietary interfaces.

%\subsection*{Introduction to ``Resources, Events and Agents''}
%%Introducing REA
%The REA ontology was introduced by William E. Mccarthy in 1982 as ``A Generalized Framework for Accounting Systems in a Shared Data Environment''. \cite{mccarthy1982rea}
%Its core idea is to "catpure the essence" for one business transaction in every aspect.
%With its basic building blocks ``events, resources and agents'' it is possible to model almost every flow of the usual acquisition-, conversion- and sales-processes with actors in- and outside a company and later restore the needed point of view at runtime.

%Expressivness...
%It has it's merits 

%%% nice for the thesis
%a little history
%Later it was extended to model future events with contracts and typification \cite{mccarthy2006reapolicy} and got accepted as a common language in Information Systems literature to describe Business Patterns (\cite{dunn2005enterpriseinfosys}, \cite{hruby2006modeldrivendesign} and \cite{hollander2000accounting}). 
%In 2007 the semantic concepts of REA where reused in an UN/CEFACT standard \cite{ISOIEC1594442007} as an intermediate language called Open-edi Business Transaction Ontology (OeBTO).
%But with it's idea to deal with most detailed data it was almost impossible to use it for real software in 1982.
%Since then technology evolved and it's patterns are used by a real world software called ``Workday'' \cite{nittler2012modernizeaccounting}.
%There was a project held at TU Wien that built a prototype ERP system on top of REA and on outcome was an API to store shared data \cite{wally2015realista}.
%This makes it possible to have a unified, vendor-agnostic storage engine even with third-party software where each department can restore it's favoured point of view.
%
%Figure \ref{fig:REA-maturity-levels} visualizes the different maturity levels for systems mentioned in this section. It shows from left to right separated information systems with explicit data exchange, then an ERP system with integrated data exchange and finally a representation of what could be achieved if REA was used to store all transaction in one information system.
%\begin{figure}
%\centering
%\caption{Different information system maturity levels to REA}
%\label{fig:REA-maturity-levels}
%\includegraphics[width=1.0\linewidth]{"../figures/Drawing 00 - Different IS maturity levels to REA"}
%\caption*{Distributed systems have their own storage and data exchange often happens manually. ERP systems often use subsystems internally. These two variants lead to duplicates and complicated data storage. REA stores all events in unified way and restores the point of view for each deparment at runtime.}
%\end{figure}
%
%\subsection*{REA by example}
%
%Figure \ref{fig:REA-simple-sample} shows an example of a sales event with delayed payment in REA: The sales event leads to a stock-flow of cookies from the company to the customer. The reciprocal payment event leads to a stock-flow of cash from customer to the company and has not yet happened. Instead there is a commitment between customer and company to pay the amount in cash in future.



%In traditional Information Systems this example would create data in a few subsystems:
%The sales department wants to benchmark sales of different products and control marketing.
%The stocks department needs to follow stock counts and has to order new items if needed.
%The finance department is responsible for internal and external financial reporting.
%The treasury has to follow unpaid invoices.
%Every department has their own needs and would generate their own data in an usual system.
%But REA proposes a data model with all information for the transaction available in one single place and every department has access to their point of view.

%OR EXTEND THIS
% to much here?
%Specifically in traditional double-entry bookkeeping these events would lead to three bookings: One entry for the revenue against a receivable and one entry for the cost of sales against output of goods. When the customer paid the receivable is cleared against cash.
%In this situation REA has it's benefits by making future payments directly visible compared to traditional accounting.
% It's needed to record additional information in cost-accounting if the revenue per transaction is of interest but in REA the collected informations belong together by nature and can later be aggregated as desired.

%\begin{figure}
%	
%	\centering
%	\caption{An example for a transaction in REA ontology}
%	\label{fig:REA-simple-sample}
%	\includegraphics[width=0.6\linewidth]{"../figures/Extended REA Example"}
%	\caption*{This example shows the sales of cookies with delayed payment in cash. Based on \cite{mccarthy2004elmo-cookie-monster}. This diagram shows only parts of the REA ontology that are in scope of this work and omits other data like VAT.}
%\end{figure}
%
%\subsection*{Financial accounting}
%
%In financial accounting the "ALE" equitation $$ Assets = Liabilities + Equity $$ has always to be true.
%This systematic is called ``ALE accounting'' because every booking is affecting one or more of it's building blocks ``Assets'', ``Liabilities'' and ``Equity''. \cite{schwaiger2015aleandrea}
%In standard double entry bookkeeping this equitation is always true because both sides of the equitation have to be booked at the same time.
%Assets belong to the company whereas Liabilities and Equity represent debts.
%Incomes and expenses are considered as in- or decreases to owners equity. \cite{horngren2006introduction}
%There are 9 different possibilities to book between these three categories as shown in Figure \ref{fig:ale-accounting---schwaiger}.
%Bookings that involve Equity in any way are booked to P\&L accounts and then forwarded to Equity when a new year is opened.

%\begin{figure}
%	\centering
%	\caption{ALE Accounting Matrix}
%	\label{fig:ale-accounting---schwaiger}
%	\includegraphics[width=0.4\linewidth]{"../figures/ALE Accounting - Schwaiger"}
%	\caption*{with 9 transaction types, source: \cite{schwaiger2015aleandrea}}.
%\end{figure}

%For better understandability all nine possible cases are listed and explained by examples:
%\begin{enumerate}
%	\item \emph{Accounting exchange on the asset side} Here an asset is exchanged with another asset. A good example for such a case is the purchase of merchandise with cash. Inventory is added while cash is being reduced.
%	\item \emph{Exchange between Assets and Liabilities} Here an asset is exchanged in return for a liability, e.g. purchase of merchandise with delayed payment. Inventory is added while the company is obliged to pay in future to the external actor and therefore also adds a liability.
%	\item \emph{Accounting exchange on the liability side} A liability is exchanged with another liability. This could be 
%\end{enumerate}

%Their meanings are as follows: (Taken from \cite{schwaiger2015aleandrea})
%\begin{enumerate}
%	\item  accounting exchange on the assets side – e.g. acquisition transaction
%	\item balance sheet extension – e.g. debt financing transaction
%	\item accounting exchange on the liabilities side – e.g. refinancing transaction
%	\item balance sheet contraction – e.g. loan redemption transaction
%	\item expenses (balance sheet contraction) – e.g. depreciation expenses
%	\item expenses (balance sheet "neutral") – e.g. provisioning
%	\item revenues (balance sheet extension) – e.g. revaluation profits
%	\item revenues (balance sheet "neutral") – e.g. unused provision release
%	\item accounting exchange on the equity side – e.g. creation of equity reserves
%\end{enumerate}


%\subsection*{REA as ALE accounting model}
%
%Although McCarthy proposed in his initial paper \cite{mccarthy1982rea} the REA ontology as suitable for financial accounting as well, he concentrated on the asset side only.
%He stated that all physical and intangible assets can be represented by resources and all kind of receivables by claims.
%Equity is identified as ``imbalance'' and it is proposed to model it as a ``stock information object''.
%Liabilities are not mentioned explicitly and it's unclear how to cope with them.
%Later on a policy layer was added in \cite{mccarthy2006reapolicy} to cover these as future events.
%Equity and liabilities are future outflows of cash or cash-equivalences in traditional accounting and core components for analysing a company's financial situation.
%They are especially in the banking sector treated as resources themselves that can be traded or leverage investments.
%As resources can not emerge spontaneous it's important to model them in the moment the obligation emerges and not when it is resolved.
%McCarthy proposes a mapping between events and income and expenses as well but it's questionable if this definition is sufficient in view of the simultaneous changes in equity.

%REA itself has following \cite{ISOIEC1594442015} its focus on business transactions between companies and takes an independent view on them.

%stock flow -> value flow

%Offen hier:
%\begin{itemize}
%	\item \cite{ISOIEC1594442015} Introduction
%	\begin{itemize}
%		\item Fokus auf Business transactions
%		\item ISO: Rea als generisches Accounting vs financial accounting
%		\item ISO: Rea als unabhängige Sprache während financial accounting interne Sicht hat
%		\item Independent view ist im accounting nicht möglich, da im financial accounting auch Geschehnisse innerhalb des Unternehmens von Relevanz sind. (0.3)
%	\end{itemize}
%	\item How to get a P\&L?
%	\item What to do with equity?
%	\item Wo ist der value exchange bei Steuern?
%	\item Was tun bei Buchungen ohne Ressourcenfluß Rückstellungen, Neubwertungen
%	\item T-Accounting
%	\item REA ist zukunftsorientiert durch Policy layer
%\end{itemize}

%W.S.A. Schwaiger enhanced the REA business transaction model to the REA ALE accounting model in \cite{schwaiger2015aleandrea} to enable proper ALE accounting.
%In this paper (present) value flows instead of stock flows where introduced to enable a value restriction like in ALE account.
%Additionally debit and credit events where introduced.


%Traditionally REA is INTER which brings issues with determing the revenue of a transaction and also leaves aside taxes and equity.


%\subsection*{Research domain}
%
%There is a lot of literature that teaches how to model business cases in REA but there is still a lack of formalization how to restore the point of view for financial accounting.
%The formalization in \cite{ISOIEC1594442015} has its focus on business transactions between companies and therefore takes an independent view on them.
%And financial accounting with double entry bookings is the final destination for almost every data that are produced over time either to give condensed performance indicators for management or to create statutory reports.
%REA is good at modelling assets as they are simply resources that belong to a company.
%But as liabilities and equity are very important concepts for financial accounting and are somehow treated as resources, there is a need for more investigation how to cope with them in REA and if the mapping for events is complete.
%In REA exist two concepts ``claim'' and ``commitment'' that seem to be redundant and need further investigation.
%These points lead to the following research question: ``What formalisms have to be defined for financial accounting in REA?''

%McCarthy argues "that the semantic modeling of accounting object systems should not include elements of double-entry bookkeeping such as debits, credits, and accounts. ...these Elements are artifacts associated with journals and ledgers. As such, they are not essential apsects of an accounting system." \cite[p. 559-560]{mccarthy1982rea}

%. \todo{ausbauen mit policy layer aber mit Blick auf Verträge}

%One main issue restoring financial accounting from REA models is that it it's not clear how to deal with Equity. Equity is handled in standard double entry bookkeeping as residual amount between Assets and Liabilities and is at the same time the total of all profits and losses over one period. It's during the economic year a residual measure that is calculated implicitly.

%REAs way of modelling complete transactions with all available details at runtime makes it mandatory for financial accounting to model Equity explicitly whenever a profit or loss is gained.
%But actually and in REAs thinking as well it is a future flow of cash to the companies shareholders and has therefore be modelled explicit every process that touches financial accounting.
%REA with it's explicit and \cite{schwaiger2015aleandrea}

%But for ALE accounting its crucial to also know what is "inside" a company, to get typical artefacts of ALE accounting, most important the balance sheet that besides assets and liabilities also includes profits or losses and equity.
%REA itself has no concept for handling these items properly in it's standardized variant.
%Anyhow W.S.A. Schwaiger enhanced it in \cite{schwaiger2015aleandrea}, it is still not evaluated that all concepts of ALE accounting are covered by this enhancement especially when there is no physical resource flow happening like provisions or stock revaluations.

%REA's approach to store information in an unified way is it worth to follow. It has many benefits against the classic storage of the same original data in distributed systems or communicating, connected  subsystems in an ERP System where each subsystem stores it's own point of view plus links to other subsystems.
%There is no practical manual for financial accounting with REA.
%It lacks a case study and definitions how to model financial bookings in REA without braking the central idea to be a generalized 

%explicit definition


%So the main problem statements is: "There is no concise definition how to cope with financial accounting and especially equity in REA ontology."


%An evaluation of the REA ALE Model based on typical bookings is

\section{Expected outcome}\label{sec:outcome}
First expected outcome is an accounting modelling concept that supports the semantics of all critical elements in accrual accounting. It will be based on REA and use UML class diagrams like in ISO/IEC 19544-4:2015 \cite{ISOIEC1594442015}.

%The expected outcome of the planned master thesis is an evaluation of typical booking cases like mentioned earlier in figure \ref{fig:ale-accounting---schwaiger}.% in Section \ref{sec:aleaccounting}.
It will be based on a case study where every shortcoming in table \ref{tab:accrualaccounting} is deeply examined with examples based on IFRS. Then requirements are defined and an approach to model these with elements of REA or new elements or attributes is proposed.

As further outcomes based on this modelling concept these artefacts are planned:
\begin{itemize}
	\item Typification: What kind of hierarchies is needed to categorize resources, events, agents and future events in accrual  accounting
	\item Mapping: What types of resources, events and agents correspond to what kind of accounts
	\item Constraints: Which constraints have to be fulfilled to enable extraction of financial accounting from REA models
\end{itemize}

This expected outcome will support the development of REA based accounting systems.


%The expected outcome of the planned master thesis is a framework for modelling phenomena of financial accounting in REA.
%It will describe needed properties of REA models to have all information for financial accounting statements based on principles from ISO standards, existing business patterns and terms from financial accounting.
%At best it will make it possible to include the ``ALE-equitation'' in REA directly.
%
%It will especially focus on:
%
%\begin{itemize}
%	\item Typification: What kind of hierarchies is needed to categorize resources, events and future events in financial accounting
%	\item Liabilities \& Equity: A guidance how to cope with liabilities and especially equity in scope of mechanics of REA 
%	\item Constraints: Which constraints have to be fulfilled to enable extraction of financial accounting from REA models
	
%\end{itemize}

%equity
%provisions
%liabilities

%it will describe how to cope with financial phenomena and what can be done implicit and what has to be defined explicit.


%\item general principles of mapping between financial accounting statements (General ledger, Customers, Suppliers...) and REA - UML Diagrams, DSL, Constraints (textual description?), Mapping between ressources, events and their types to general accounting.

\section{Methodology and approach}

The methodological approach will in analogy to Geert's paper on design science research \cite{geerts2010designsience} consist of these phases:

%\subsection{Problem Identification and Motivation}
\begin{enumerate}
	\item Problem Identification and Motivation: At first requirements of accrual accounting and REA are presented and summarized as outcome of a literature research.
	\item Define the Objectives of a Solution:
	Then these requirements are compared and further actions are defined in analogy to table \ref{tab:accrualaccounting}. Properties and attributes needed for accrual accounting are identified (e.g. ageing of liabilities, probability of provisions).
	Objectives are artifacts depicted in Section \ref{sec:outcome}.
	\item Design and Development: For every requirement a way to model it in an accrual accounting model either with existing elements of REA or newly introduced elements, attributes or constraints is proposed.
	This will include a proposed mapping between elements and eventually add new types for resources, events or agents.
	\item Demonstration: A set of typical real-life booking-cases that cover all sections of the ALE accounting matrix in figure \ref{fig:ale-accounting---schwaiger} is defined and analysed in the context of REA.
	Then each case is designed as instance of a REA accounting model based on the earlier developed concept.
	\item Evaluation: Each case is tested if it holds all the information required for the accrual accounting artifacts according to the objectives.
	After the case studies are finished, a conclusion is done whether or not it is possible to implement all elements of accrual accounting in the REA accrual accounting model with its existing enhancements.
%	If it's possible, the typifications, constraints and methodologies to cope with liabilities and equity used to model the different booking cases are summarized.
\end{enumerate}

%At first a thorough definition of all REA related concepts will be made. The most important sources will be papers written by William E. McCarthy (e.g. \cite{mccarthy1982rea}), papers written by Schwaiger (e.g. \cite{schwaiger2015aleandrea}), the ISO 15944:4 standard in it's version from 2015 \cite{ISOIEC1594442015} and books about Information Systems design with REA (\cite{dunn2005enterpriseinfosys}, \cite{hruby2006modeldrivendesign} and \cite{hollander2000accounting}). The outcome will be an in deep description of all used terms especially with focus on financial accounting.
 

%\subsection{Design of a accrual modelling concept}

%Based on elements in table \ref{tab:accrualaccounting} needed properties for accrual accounting are identified and discussed in context of REA.
%For each of these elements a 



%A set of typical real-life booking-cases that cover all sections of the ALE accounting matrix in figure \ref{fig:ale-accounting---schwaiger} is defined and analysed in the context of REA.
%Then each case is designed as REA ALE accounting model.
%A transformation from these models to lines in a journal for financial accounting is done to show that all information needed for ALE accounting are expressed in these REA instances.
%This will start with basic examples that can be easy modelled in REA for purchases, sales and don't include any future events.
%Later more complex situations as delayed payments, provisions and financial instruments will be analysed.
%At last transactions from and to Equity will be discussed.

%\subsection{Evaluation and conclusion}

%In the phase of evaluation a set of typical real-life booking-cases that cover all sections of the ALE accounting matrix in figure \ref{fig:ale-accounting---schwaiger} is defined and analysed in the context of REA.
%Then each case is designed as REA ALE accounting model based on the earlier developed concept.
%A transformation from these models to lines in a journal for financial accounting is done to show that all information needed for ALE accounting are expressed in these REA instances.
%
%This will start with basic examples that can be easy modelled in REA for purchases, sales and don't include any future events.
%Later more complex situations as delayed payments, provisions and financial instruments will be analysed.
%At last transactions from and to Equity will be discussed.

%Each case is tested if it holds all the information needed for the accrual accounting artifacts mentioned in table \ref{sec:problem}.



\begin{figure}
	\centering
	\caption{ALE Accounting Matrix}
	\label{fig:ale-accounting---schwaiger}
	\includegraphics[width=0.4\linewidth]{"../figures/ALE Accounting - Schwaiger"}
	\caption*{with 9 transaction types, source: \cite{schwaiger2015aleandrea}}
\end{figure}

%\subsection*{ Generalized framework}
%
%Based on the findings in the case studies general rules for the planned framework are extracted and presented in a formalized way.
%This will end in a set of type hierarchies and constraints that work on these type hierarchies.

%\subsection*{Conclusion}
%\todo{say something!}

\section{State of the art}

In addition to Papers by McCarthy and Geerts there is some literature that focuses on the datamodel and business patterns (\cite{dunn2005enterpriseinfosys}, \cite{hruby2006modeldrivendesign} and \cite{hollander2000accounting}).
In 2007 the semantic concepts of REA where reused in an UN/CEFACT standard \cite{ISOIEC1594442015} as an intermediate language called Open-edi Business Transaction Ontology (OeBTO).
Right now REA seems to be established as an intermediate language.  

There was a study group at BIG TU Wien that lead to some papers (e.g. \cite{wally2015modeldriven}) about REA.
It's outcome was a layered prototype of a cloud-based REA-based ERP system with a REA-based storage API.

There is an online ERP system for HR and Finance called "Workday" publicly available and it's claimed to be based on the REA ontology (\cite{nittler2012modernizeaccounting} and \cite{howlett2007workdayfinancials}).

%
%\todo{SHINY}

%Altough P. Hruby has added some modeling base for accounts \cite[p.180ff]{hruby2006modeldrivendesign}

\section{Relevance}

The REA-ontology is teached in the curriculum Bussines Informatics and there is research about this at TU Wien as well.
%Further areas touched by this work are foremost Business Modeling and e-commerce with REA's value-chain model.
It combines several techniques and methodologies discussed in different courses listed beneath:

\begin{itemize}
	%\item E-Commerce (value chain)
	\item Accounting and Finance: The topic of the thesis is clearly settled in this academic field
	\item Model Engineering: Techniques from this course a used to describe properties for financial REA models
	\item E-Commerce: REA touches the field of electronic data interchange (EDI) 
	\item Enterprise Information Systems: This thesis discusses properties of informations systems as well
\end{itemize}





