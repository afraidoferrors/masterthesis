% !TeX spellcheck = en_GB

%this is needed to leave out the top level numbering by chapters in this proposal
\renewcommand{\thefigure}{\arabic{figure}}
%this controls the width of long captions
\captionsetup{width=0.8\textwidth}



%\todo{Created / Printed: \today}

\section*{Problem definition}

\subsection*{Drawbacks of traditional Information Systems}

%Problem statement - existing accounting systems are cumbersome
Business Information Systems are commonly used to store data that are shared between different departments like HR, Sales, Production, Accounting etc.
If a company uses separate systems this means often incompatible ``data-islands'' and that leads to a lot of adapting tasks to get data from one system, sanitize and import them into another system.
Data in separated systems are duplicated by design and data synchronization tasks have to be actively maintained.
ERP systems only solve parts of this problems because they usually consist of more or less integrated subsystems that exchange data internally.
This architecture leads to duplicates again and if data get corrupt it's a painful task to restore a working state.

%%% nice for the thesis, to much for the proposal
%\cite[pages 2-6]{dunn2005enterpriseinfosys} argues that specialized departments inside a company tend to build their own ecosystem of applications after their needs that don't interoperate with systems from other departments by default.
%If not taken care processes are likely to focus only on the ``own'' department instead of being integrated to the companies holistic view.
%This leads almost always to non-integrated systems that have to run besides the ERP system, receive or send data and there is always a need to maintain ETL-tasks for proprietary interfaces.

\subsection*{Benefits of ``Ressources, Events and Agents''}
%Introducing REA
The REA ontology was introduced by William E. Mccarthy in 1982 as ``A Generalized Framework for Accounting Systems in a Shared Data Environment''. \cite{mccarthy1982rea}
It's core idea is to model every aspect for every department that belongs to one business transaction.
With it's basic building blocks ``events, resources and agents'' it's possible to model almost every flow of the usual acquisition-, conversion- and sales-processes with actors in- and outside a company and later restore the needed point of view at runtime.
%%% nice for the thesis
%a little history
%Later it was extended to model future events with contracts and typification \cite{mccarthy2006reapolicy} and got accepted as a common language in Information Systems literature to describe Business Patterns (\cite{dunn2005enterpriseinfosys}, \cite{hruby2006modeldrivendesign} and \cite{hollander2000accounting}). 
%In 2007 the semantic concepts of REA where reused in an UN/CEFACT standard \cite{ISOIEC1594442007} as an intermediate language called Open-edi Business Transaction Ontology (OeBTO).
%But with it's idea to deal with most detailed data it was almost impossible to use it for real software in 1982.
%Since then technology evolved and it's patterns are used by a real world software called ``Workday'' \cite{nittler2012modernizeaccounting}.
%There was a project held at TU Wien that built a prototype ERP system on top of REA and on outcome was an API to store shared data \cite{wally2015realista}.
This makes it possible to have a unified, vendor-agnostic storage engine even with third-party software where each department can restore it's favoured point of view.

Figure \ref{fig:REA-maturity-levels} visualizes the different maturity levels for systems mentioned in this section. It shows from left to right separated information systems with explicit data exchange, then an ERP system with integrated data exchange and finally a representation of what could be achieved if REA was used to store all transaction in one information system.
\begin{figure}
\centering
\caption{Different information system maturity levels to REA}
\label{fig:REA-maturity-levels}
\includegraphics[width=1.0\linewidth]{"../figures/Drawing 00 - Different IS maturity levels to REA"}
\caption*{Distributed systems have their own storage and data exchange often happens manually. ERP systems often use subsystems internally. These two variants lead to duplicates and complicated data storage. REA stores all events in unified way and restores the point of view for each deparment at runtime.}
\end{figure}

\subsection*{REA by example}

Figure \ref{fig:REA-simple-sample} shows an example of a sales event with delayed payment in REA: The sales event leads to a stock-flow of cookies from the company to the customer. The reciprocal payment event leads to a stock-flow of cash from customer to the company and has not yet happened. Instead there is a commitment between customer and company to pay the amount in cash in future.

\begin{figure}

\centering
\caption{An example for a transaction in REA ontology}
\label{fig:REA-simple-sample}
\includegraphics[width=0.6\linewidth]{"../figures/Extended REA Example"}
\caption*{This example shows the sales of cookies with delayed payment in cash. Based on \cite{mccarthy2004elmo-cookie-monster}. This diagram shows only parts of the REA ontology that are in scope of this work and omits other data like VAT.}
\end{figure}

In traditional Information Systems this example would create data in a few subsystems:
The sales department wants to benchmark sales of different products and control marketing.
The stocks department needs to follow stock counts and has to order new items if needed.
The finance department is responsible for internal and external financial reporting.
The treasury has to follow unpaid invoices.
Every department has their own needs and would generate their own data in an usual system.
But REA proposes a data model with all information for the transaction available in one single place and every department has access to their point of view.

%OR EXTEND THIS
% to much here?
%Specifically in traditional double-entry bookkeeping these events would lead to three bookings: One entry for the revenue against a receivable and one entry for the cost of sales against output of goods. When the customer paid the receivable is cleared against cash. It's needed to record additional information in cost-accounting if the revenue per transaction is of interest but in REA the collected informations belong together by nature and can later be aggregated as desired.

\subsection*{Financial accounting}

In financial accounting the "ALE" equitation $$ Assets = Liabilities + Equity $$ has always to be true.
This systematic is called ``ALE accounting'' because every booking is affecting one or more of it's building blocks ``Assets'', ``Liabilities'' and ``Equity''. \cite{schwaiger2015aleandrea}
In standard double entry bookkeeping this equitation is always true because both sides of the equitation have to be booked at the same time.
Assets belong to the company whereas Liabilities and Equity represent debts.
Incomes and expenses are considered as in- or decreases to owners equity. \cite{horngren2006introduction}
There are 9 different possibilities to book between these three categories as shown in Figure \ref{fig:ale-accounting---schwaiger}.
%Bookings that involve Equity in any way are booked to P\&L accounts and then forwarded to Equity when a new year is opened.

\begin{figure}
	\centering
	\caption{ALE Accounting Matrix}
	\label{fig:ale-accounting---schwaiger}
	\includegraphics[width=0.4\linewidth]{"../figures/ALE Accounting - Schwaiger"}
	\caption*{with 9 transaction types, source: \cite{schwaiger2015aleandrea}}.
\end{figure}

\subsection*{REA and ALE accounting}

Although McCarthy proposed in his initial paper \cite{mccarthy1982rea} the REA ontology as suitable for financial accounting as well, he concentrated on the asset side only.
He stated that all physical and intangible assets can be represented by resources and all kind of receivables by claims.
Equity is identified as ``imbalance'' and it is proposed to model it as a ``stock information object''.
Liabilities are not mentioned explicitly and it's unclear how to cope with them.
Equity and liabilities are future outflows of cash or cash-equivalences in traditional accounting and as resources can not emerge spontaneous it's important to model it in the moment the obligation emerges and not when it is resolved.
McCarthy proposes a mapping between events and income and expenses as well but it's questionable if this definition is sufficient in view of the simultaneous changes in equity.

\subsection*{Research question}

There is a lot of literature that teaches how to model business cases in REA but there is still a lack of formalization how to restore the point of view for financial accounting.
%And financial accounting with double entry bookings is the final destination for almost every data that are produced over time either to give condensed performance indicators for management or to create statutory reports.
REA is good at modelling assets as they are simply resources that belong to a company.
But there is a need for more investigation how to cope with liabilities and equities in REA and if the mapping for events is complete.
In REA exist two concepts ``claim'' and ``commitment'' that seem to be redundant and need further investigation.
These points lead to the following research question: ``What formalisms have to be defined for financial accounting in REA?''

%McCarthy argues "that the semantic modeling of accounting object systems should not include elements of double-entry bookkeeping such as debits, credits, and accounts. ...these Elements are artifacts associated with journals and ledgers. As such, they are not essential apsects of an accounting system." \cite[p. 559-560]{mccarthy1982rea}

%. \todo{ausbauen mit policy layer aber mit Blick auf Verträge}

%One main issue restoring financial accounting from REA models is that it it's not clear how to deal with Equity. Equity is handled in standard double entry bookkeeping as residual amount between Assets and Liabilities and is at the same time the total of all profits and losses over one period. It's during the economic year a residual measure that is calculated implicitly.

%REAs way of modelling complete transactions with all available details at runtime makes it mandatory for financial accounting to model Equity explicitly whenever a profit or loss is gained.
%But actually and in REAs thinking as well it is a future flow of cash to the companies shareholders and has therefore be modelled explicit every process that touches financial accounting.
%REA with it's explicit and \cite{schwaiger2015aleandrea}



%REA's approach to store information in an unified way is it worth to follow. It has many benefits against the classic storage of the same original data in distributed systems or communicating, connected  subsystems in an ERP System where each subsystem stores it's own point of view plus links to other subsystems.
%There is no practical manual for financial accounting with REA.
%It lacks a case study and definitions how to model financial bookings in REA without braking the central idea to be a generalized 

%explicit definition


%So the main problem statements is: "There is no concise definition how to cope with financial accounting and especially equity in REA ontology."


\section*{Expected outcome}

The expected outcome of the planned master thesis is a framework for modelling phenomena of financial accounting in REA.
It will describe needed properties of REA models to have all information for financial accounting statements based on principles from ISO standards, existing business patterns and terms from financial accounting.
At best it will make it possible to include the ``ALE-equitation'' in REA directly.

It will especially focus on:

\begin{itemize}
	\item Typification: What kind of hierarchies is needed to categorize resources, events and future events in financial accounting
	\item Liabilties: A guidance how to cope with liabilities - especially equity - in scope of mechanics of REA 
	\item Constraints: Which constraints have to be fulfilled to enable extraction of financial accounting from REA models
	
\end{itemize}

%equity
%provisions
%liabilities

%it will describe how to cope with financial phenomena and what can be done implicit and what has to be defined explicit.


%\item general principles of mapping between financial accounting statements (General ledger, Customers, Suppliers...) and REA - UML Diagrams, DSL, Constraints (textual description?), Mapping between ressources, events and their types to general accounting.

\section*{Methodology and approach}

The methodological approach consists of three parts.

\subsection*{Literature research and identification of relevant parameters}

At first a thorough definition of all REA related concepts will be made. The most important sources will be papers written bei William E. McCarthy (e.g. \cite{mccarthy1982rea}), the ISO 15944:4 standard in it's version from 2007 \cite{ISOIEC1594442007} and books about Information Systems design with REA (\cite{dunn2005enterpriseinfosys}, \cite{hruby2006modeldrivendesign} and \cite{hollander2000accounting}). The outcome will be an in deep description of all used terms especially with focus on financial accounting.

\subsection*{ Case studies for typical booking cases}

A set of typical booking-cases is defined and then a mapping in REA with focus only on financial accounting is done for each case.
This will start with basic examples that can be easy modelled in REA for purchases, sales and don't include any future events.
Later more complex situations as delayed payments and provisions will be analysed.
At last transactions from and to Equity will be discussed.

\subsection*{ Generalized framework}

Based on the findings in the case studies general rules for the planned framework are extracted and presented in a formalized way.
This will end in a set of type hierarchies and constraints that work on these type hierarchies.

%\subsection*{Conclusion}
%\todo{say something!}

\section*{State of the art}

In addition to Papers by McCarthy and Geerts there is some literature that focuses on the datamodel and business patterns (\cite{dunn2005enterpriseinfosys}, \cite{hruby2006modeldrivendesign} and \cite{hollander2000accounting}).
In 2007 the semantic concepts of REA where reused in an UN/CEFACT standard \cite{ISOIEC1594442007} as an intermediate language called Open-edi Business Transaction Ontology (OeBTO).
Right now REA seems to be established as an intermediate language.  

There was a study group at TU Wien (See \url{http://www.big.tuwien.ac.at/projects/29}) that lead to some papers (e.g. \cite{wally2015modeldriven}) about REA.
It's outcome was a layered prototype of a cloud-based REA-based ERP system with a REA-based storage API.

There is an online ERP system for HR and Finance called "Workday" publicly available and it's claimed to be based on the REA ontology (\cite{nittler2012modernizeaccounting} and \cite{howlett2007workdayfinancials}).

%Altough P. Hruby has added some modeling base for accounts \cite[p.180ff]{hruby2006modeldrivendesign}

\section*{Relevance}

The REA-ontology is teached in the curriculum Bussines Informatics and there is research about this at TU Wien as well.
%Further areas touched by this work are foremost Business Modeling and e-commerce with REA's value-chain model.
It combines several techniques and methodologies discussed in different courses listed beneath:

\begin{itemize}
	%\item E-Commerce (value chain)
	\item Accounting and Finance: The topic of the thesis is clearly settled in this academic field
	\item Model Engineering: Techniques from this course a used to describe properties for financial REA models
	\item E-Commerce: REA touches the field of electronic data interchange (EDI) 
	\item Enterprise Information Systems: This thesis discusses properties of informations systems as well
\end{itemize}





